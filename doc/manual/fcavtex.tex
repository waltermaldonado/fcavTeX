%% unespfcav.tex, v<VERSION> waltermaldonado
%% Copyright 2015 by Walter Maldonado Jr 
%%
%% This work may be distributed and/or modified under the
%% conditions of the LaTeX Project Public License, either version 1.3
%% of this license or (at your option) any later version.
%% The latest version of this license is in
%%   http://www.latex-project.org/lppl.txt
%% and version 1.3 or later is part of all distributions of LaTeX
%% version 2005/12/01 or later.
%%
%% This work has the LPPL maintenance status `maintained'.
%%
%% The Current Maintainer of this work is Walter Maldonado Jr
%%
%% Creator and original mantainer: Walter Maldonado Jr <walter@agroestat.com.br>

\documentclass[a4paper]{ltxdoc}
\usepackage{lmodern}			% Usa a fonte Latin Modern			
\usepackage[T1]{fontenc}		% seleção de códigos de fonte.
\usepackage[utf8]{inputenc}		% determina a codificação utiizada (conversão automática dos acentos)
\usepackage{hyperref}  			% controla a formação do índice
\usepackage{parskip}			% espaçamento entre os parágrafos
\usepackage{microtype} 			% para melhorias de justificação
\usepackage{morefloats}			% permite mais floats


% Babel e ajustes
\usepackage[brazil]{babel}		% idiomas
\addto\captionsbrazil{
    %% ajusta nomes padroes do babel
    \renewcommand{\bibname}{Refer\^encias}
    \renewcommand{\indexname}{\'Indice}
    \renewcommand{\listfigurename}{Lista de ilustra\c{c}\~{o}es}
    \renewcommand{\listtablename}{Lista de tabelas}
    %% ajusta nomes usados com a macro \autoref
    \renewcommand{\pageautorefname}{p\'agina}
    \renewcommand{\sectionautorefname}{se{\c c}\~ao}
    \renewcommand{\subsectionautorefname}{subse{\c c}\~ao}
    \renewcommand{\paragraphautorefname}{par\'agrafo}
    \renewcommand{\subsubsectionautorefname}{subse{\c c}\~ao}
    \renewcommand{\paragraphautorefname}{subse{\c c}\~ao}
}  

\title{\textbf{A classe \textsf{fcavTeX}} \\ \Large{Teses e dissertações \\ Faculdade de Ciências Agrárias e Veterinárias de Jaboticabal}}

%   \thanks{Este documento
%   se referete ao \textsf{abntex2} versão \fileversion,
%   de \filedate.}
  
\author{Walter Maldonado Jr\\walter@agroestat.com.br} 

\date{\today, v<VERSION>}

\hypersetup{
		pdftitle={A classe fcavTeX},
		pdfauthor={Walter Maldonado Jr},
    	pdfsubject={Teses e dissertações da Faculdade de Ciências Agrárias e Veterinárias de Jaboticabal}, 
    	pdfkeywords={FCAV}{UNESP}{trabalho acadêmico}{dissertação}{tese}, 
		pdfproducer={Walter Maldonado Jr -- walter@agroestat.com.br}, 	% producer of the document
	    pdfcreator={LaTeX with fcavTeX},
    	colorlinks=true,
    	linkcolor=blue,
    	citecolor=blue,
		urlcolor=blue
}

\EnableCrossrefs
\CodelineIndex
\RecordChanges

\changes{v1.0}{2015/06/27}{Versão inicial}

\usepackage{xcolor}
\usepackage{listings}

\lstset
{
    language=[LaTeX]TeX,
    breaklines=true,
    basicstyle=\tt\scriptsize,
    keywordstyle=\color{blue},
    identifierstyle=\color{black},
    extendedchars=true,
    literate={á}{{\'a}}1 {ã}{{\~a}}1 {é}{{\'e}}1 {ó}{{\'o}}1 {ç}{{\c{c}}}1 {í}{{\'i}}1 {Ç}{{\c{C}}}1 {Ã}{{\~A}}1 {Â}{{\^A}}1 {ô}{{\^o}}1 {õ}{{\~o}}1,
}

\begin{document}


\maketitle

\begin{abstract}
A formatação de um trabalho
acadêmico é sempre uma tarefa árdua, mecânica e cansativa. Quando
nos deparamos com um trabalho extenso e com diversas referências bibliográficas,
semanas de trabalho se vão. A classe fcavTeX poupa o usuário de horas perdidas
em frente a um editor de texto limitado que não irá garantir a padronização do 
seu documento. Com um conhecimento básico de \LaTeX \ é possível escrever todo a
sua tese ou dissertação sem precisar se preocupar com tamanho de fonte, espaçamento, etc.
Quantos espaços após o título eu devo dar? Será que esse título é em negrito? São perguntas
que não merecem o tempo de um pesquisador.
\end{abstract}

\tableofcontents

\listoftables
 
% ------
\section{Introdução}
% ------

Eu adoro tecnologia. A cada dia que eu trabalho com ela fico mais fascinado com todas as possiblilidades
e todos os recursos que ela nos oferece. É fascinante. Se compararmos algumas tarefas que executamos hoje com a maneira
com que eram desenvolvidas dez anos atrás veremos que há um progresso imenso. Tomem como exemplo os \emph{smartphones} 
e como ficou fácil gerenciar a nossa agenda de compromissos, que além de tudo nos alerta quando esquecemos de algo,
ao nosso alcance 24h por dia.

E nós ainda estamos escrevendo nossos trabalho acadêmicos como escrevíamos há 25 anos. Escrevemos um título, selecionamos
e mudamos os atributos de fonte. Quando muito, temos estílos pré-definidos para essa tarefa, o que ajuda um pouco mas
não é suficiente. Digo que não é suficiente pois, se estamos fazendo um trabalho acadêmico, com certeza estamos seguindo 
diversas regras que nos são impostas. Regras essas que \emph{todos} os alunos deverão seguir.

Pois bem, quando estava decidindo como escreveria minha tese de doutorado, tais pensamentos me vieram à mente. Depois
de um tempo de reflexão cheguei à seguinte conclusão, ``Irei escrever minha tese utilizando o \LaTeX!!''. Dessa maneira,
seria possível que, além de garantir um nível de qualidade superior ao meu trabalho, poderia transformar os padrões que
desenvolveria em uma classe para que todos os alunos a pudessem utilizar e economizar muito tempo.

E aqui está a fcavTeX. Espero que seja útil e me coloco a disposição para questionamentos e para que possamos melhorá-la.
Tenho certeza de que, com um conhecimento básico de \LaTeX, muito tempo poderá ser economizado e a qualidade estética
dos trabalhos de nossa universidade será melhorada consideravelmente.

\section{Exemplo de utlilização} % (fold)
\label{sec:exemplo_de_utliliza_o}
O uso da classe é extremamente intuitivo. Os comandos são auto-decritivos e basta trocar o conteúdo entre as chaves dentro
do seu arquivo .tex. Atualmente, somente a estrutura de tese em capítulos é suportada, mas o modelo convencional também será
incluído. Todos os recursos apresentados pela classe estão em conformidade com as normas da universidade, que podem ser conferidas 
\href{http://www.fcav.unesp.br/Home/posgraduacao/normas_disss_tese.pdf}{aqui}. Segue o código do exemplo.

\begin{lstlisting}
    \documentclass{unespfcav}

    \begin{document}

    \titulo{ESTIMATIVA DA PRODUÇÃO DE CITROS USANDO IMAGENS DIGITAIS}
    \tituloingles{Citrus yield estimation using digital images}
    \autor{Walter Maldonado Jr}
    \orientador{Prof. Dr. José Carlos Barbosa}
    \qualificacaoautor{Engenheiro agrônomo}
    \instituicao{UNIVERSIDADE ESTADUAL PAULISTA - UNESP\par CÂMPUS DE JABOTICABAL}
    \tipodoc{Tese}
    \titulopretendido{Doutor}
    \programa{Agronomia (Produção Vegetal)}
    \ano{2015}

    \capa
    \folhaderosto
    \fichacatalografica
    \certificadodeaprovacao
    \dadoscurriculares{Dados curriculares aqui!}
    \epigrafe{Epígrafe}
    \dedicatoria{Dedicatória}
    \agradecimentos{Agradecimentos aqui!}
    \sumario
    \include{resumo}
    \abstract{}{}
    \listadetabelas
    \listadefiguras

    \corpodotextoemcapitulos

    \include{cap1}

    \end{document}
\end{lstlisting}

O comando \emph{include} chama o arquivo \emph{cap1.tex}, que deverá ter o seguinte formato para que todas as citações feitas com 
os comandos \emph{\textbackslash cite} possam gerar a lista de referências da maneira correta.

\begin{lstlisting}
    \chapter{Considerações gerais}
    \section{Introdução}

    Texto texto texto texto.

    \section{Revisão de Literatura}

    Primeiro parágrado da seção Revisão de Literatura. Texto texto texto.

    Segundo parágrafo.

    \bibliography{meuarquivobibtex.bib}

\end{lstlisting}

% section exemplo_de_utliliza_o (end)


\end{document}
